\documentclass[10pt,letterpaper]{article}
\usepackage[latin1]{inputenc}
\usepackage{amsmath}
\usepackage{amsfonts}
\usepackage{amssymb}
\usepackage{graphicx}
\author{Jose Lopez}
\begin{document}
	
{\Large 		This is quite a mess as I just copied it out of the book ``A dictionary of angels, including the fallen'' by Gustav Davidson. A fairly scholarly work but the important part is the Introduction which is below, to show not only what he has said but also the contradictions and unknowns he has revealed that as an author I'm more than happy to exploit. As I have of course.\\
	
	}
	
	Introduction
	Some years ago when I started "collecting" angels as a literary diversion, it was certainly with no
	thought of serving as their archivist, biographer, and finally as their lexicographer. Such an idea
	did not occur to me-indeed, could not have occurred to me-until I had corralled a sufficient
	number of the heavenly denizens to make a dictionary of them feasible.
	At first I thought that angels, named angels, were to be found only in theBible. I soon learned
	that, on the contrary, the Bible was the last place to look for them. True, angels are mentioned
	frequently enough in both the Old and New Testaments, but they are not named, save in two or
	three instances. Virtually all the named angels in this compilation are culled from sources outside
	Scripture.'
	Of the books in the New Testament, while the Synoptic Gospels and the Pauline Epistles
	have been longtime favorites of mine, the book of Revelation always held a particular fascination
	for me, mainly because, I believe, ofits apocalyptic imagery and involvenlent with angels. I read
	the book often. But one day, as I was leafing through its pages, my eye was arrested by verse 2,
	chapter 8 :
	And I saw the seven angels who stand before God;
	And to them were given seven trumpets.
	I laid the book aside and asked myself: who are these sevcn holy ollcs that stand before God?
	Has any biblical scholar identified them? Are they of the order of seraphim, cherubim, princi-
	palities, powers? And are they always the sallle seven who cnjoy the privilege and ellli~lence of
	closest proximity to the throne of Glory? And why seven? Were the seven planets the proto-
	type? Or did the notion derive froin the well-known chapter in Ezekiel 9: 2-1 1 which gives a
	terrifying picture of six "men" and a seventh "clothed in linen" whom God summoned to
	Jerusalem to "slay without pity"? Challenging, even intimidating, qhestions and ones that, I
	felt, ought not to be left unanswered. Meantime, the pursuit led me down many a heavenly
	brook. Over the years it served to unlock realms of gold I never suspected existed in Heaven or
	on earth.
	Of the seven Revelation angels I had no difficulty in establishing the identity of three:
	Michael and Gabriel (in Scripture) and Raphael (in The Book of Tobit). The last-named angel,
	by a happy chance, identifies himself: "I am Raphael," he discloses to his young charge Toby,
	"one of the seven angels who stand and enter before the glory of the Lord." No declaration
	could be more authoritative or conclusive. And so, with three of the seven angels identified, the
	problein was to bring to light the remaining four.
	1. The Koran names seven angels: Gabriel, Michael, Iblis or Eblis, chiefji~l~l
	in Arabian mythology, counterpart
	of the Judaean-Christian Satan; Malcc or Malik, principal angel ofHcll; the two fallen angcls, Harut and Mariit; and
	Malaku '1-maut, angel of death, identified as Azrael. Contrary to popular belief and accreditation, thc Koran docs
	not name Israfel, lord of the resurrection trumpet.
	
	
	[x]
	INTRODUCTION
	I remembered reading somewhere of an angel called Uriel and that he was a "regent of the
	sun." He seemed a likely candidate. I was confirmed in this feeling when I came upon Uriel in
	Paradise Lost (1 11,648 seq.) and found the archfiend himself providing warrant : "him Satan thus
	accosts./Uriel, for thou of those seav'n spirits that stand/In sight of God's high Throne,
	gloriously bright," etc. Poe's Israfel, "Whose heart-strings are a lute," was (or is) an Islamic
	angel,' and I wondered ifthat fact might rule him out. Then there was Longfellow's Sandalphon.
	In the poem by that name, Longfellow described Sandalphon as the "Angel of Glory, Angel of
	Prayer." A great angel, certainly: but, again, was he of an eminence sufficiently exalted to entitle
	him to "enter before the glory of the Lord"? That was the question. Vondel's Lucifer, Heywood's
	The Hierarchy ofthe Blessid Angels, Milton's Paradise Lost, Dryden's State oflnnocence, Klopstock's
	The Messiah-all these works yielded a considerable quantity of the celestial spirits, some in the
	top echelons, like Abdiel, Ithuriel, Uzziel, Zephon; but I had no way of telling whether any
	of them qualified. Surely, I comforted myself, there must be some source where the answer
	could be found. Actually there were a number of such sources. I had only to reach out my
	hand for books in my own library. Instead, in my then state of pneumatic innocence, I looked
	far afield.
	Since I was unacquainted at the time with anyone versed in angel lore, I decided to enter
	into correspondence with scholars and theologians who might help me. I picked half a dozen
	names at random from the faculty lists of local universities, seminaries, and yeshivas. I put the
	question squarely to them. The responses were a long time coming and hardly satisfjling.
	Not in my competence" was the way one biblical exegete put it. Another referred me to the
	minister of a Swedenborgian church in West Germany. From others I heard n o h g . But one
	rather noted maskil came through handsomely with two sets of seven, each leading off with the
	familiar m o (Michael, Gabriel, Raphael), thus:
	4 4
	First List
	Second List
	Michael
	Gabriel
	Raphael Michael
	Gabriel
	Raphael
	Uriel
	Raguel
	Saraqael
	Remiel (or Camael) Anael (Haniel)
	Zadkiel
	Orifiel
	Uzziel (or Sidriel)
	I now had iiot oilly the seven ailgels I had beell looking for but a choice of seven; and, in
	
\begin{quote}
	{\small 	2. Not a Koranic angel, as Poe mistakenly makes him out to be. Israfel is not mentioned in the Koran, and
	Poe's quotation from it must derive, presumably, from a hadith (traditional sayin attributed to the Prophet)
	or from "Preliminary Discourse," George Sale's long introductory essay to his trans ation of the Koran. Scholars
	have pointed out that references to Israfel and tributes to him as the Angel of Music in Arabic lore were known to
	Poe as occurring in the works of the French poet, de Btranger (whom Poe quotes). and the Irish poet, Thomas
	Moore.}
\end{quote}
	
	
	INTRODUCTION
	[xi]
	addition, the ilailles of angels I had not heard of b e f ~ r e In
	. ~ the course of further correspondence
	I was apprised of a branch of extracv~onical writings new to me: pseudepigrapha, particularly
	the three Elloch books, a veritable treasure-trove! Enoch I or the Book ofEnoch (also called the
	Ethiopic Elloch, froin the fact that the earliest version or recerisioil of the book was found in
	Abyssinia) was the most readily available. It literally rioted in angel names-many of them, as
	I quickly discovered, duplications or corruptions of other names.
	What were Enoch's sources? Did the patriarch (or whoever the author was to whom the
	Elloch books have been attributed) draw on his own lively imagination? (Certainly the 12-winged
	kalkydri and phoenixes were his invention.) Did he conjure his angels from the "four hinges of
	the spirit world?" Or did they come to him, as they have and still do to initiates, after a special,
	illystical concentration-a gift of grace, a charisma? I left that an open question, for the time being.
	The Enoch books led me on to related hierological sources and texts : apocalyptic, cabalistic,
	Talmudic, gnostic, patristic, Merkabah (Jewish mystic), and ultimately to the grimoires, those
	black magic manuals, repositories of curious, forbiddell, and by now well-nigh forgotten lore.
	In them, invocations, adjurations, and exorcisms were spelt out in full, often grossest detail, and
	addressed to spirits bearing the most outlandish names.The Church was not slow in pronouncing
	its curse on these rituals, although the authorship of one of the most diabolic of them was
	credited (without warrant, it is true) to a pope, Honorius the Third, who reigned during the
	years 1216-1227. The work is titled The Grimoire ofHonorius the Great, and made its first appear-
	ance in 1629, some 400 years after the death of its reputed author. Arthur Edward Waite, author
	of The Book ofCeremonia1 Magic, cites the grimoire as "a malicious and somewhat clever imposture,
	which was undeniably calculated to deceive ignorant persons of its period who may have been
	magically inclined, more especially ignorant priests, since it pretends to convey the express
	sanction of the Apostolical Seat for the operations of infernal magic and necromancy."
	All these goetic tracts yielded a boundless profusion of angels (and demons), and I soon had
	more of the fluttering creatures than I knew what to do with. In order to keep my work within
	sizable limits, I started weeding out (Heaven forgive me!) what I considered to be the less
	important names, or the ones about which little or no data could be found.
	At this stage of the quest I was literally bedeviled by angels. They stalked and leaguered me,
	by night and day. I could not tell the evil from the good, demons from daevas, satans from sera-
	phim; nor (to quote from a poem composed at the time) "if that world I could not hope to
	prove,/Flanhg with heavenly beasts, holy and grim,/Was any less real than that in which I
	moved." I moved, indeed, in a twilight zone of tall presences, through enchanted forests lit
	with the sinister splendor of fallen divinities; of aeons and archons, peris and paracletes, elohim
	and avatars. I felt somewhat like Dante, in the opening canto of The Divine Comedy, when,
	midway upon the journey of his life, he found himself astray in a dusky wood. O r like some
	knight of old, ready to try conclusions with any adversary, real or fancied. I reinember one occa-
	sion-it was winter and getting dark-returning hotne from a neighboring farm. I had cut
	
\begin{quote}
		{\small 3. Subsequently, in other lists of the seven (Enoch I, Ecdra 11, etc.), I came upon the names of the following
	angels: Jophiel, Jererniel, Pravuil, Salathiel, Sarid, Zachariel, and Zaphiel.
}	
\end{quote}
	
	[xii]
	INTRODUCTION
	across an unfamiliar field. Suddenly a nightmarish shape loomed up in front of me, barring my
	progress. After a paralyzing nloment I managed to-Kght lily way past the phantom. The next
	morning I could not be sure (no more than Jacob was, when he wrestled with his dark antagonist
	at Peniel) whether I had encountered a ghost, an angel, a demon, or God. There were other such
	moments and other such encounters, when I passed from terror to trance, from intimations of
	realms unguessed at to the uneasy conviction that, beyond the reach of our senses, beyond the
	arch of all our experience sacred and profane, there was only-to use an expression of Paul's
	in I Timothy 4-"fable and endless genealogy."
	Logic, I felt, was my only safe anchor 111 reality; but if, as Walter Nigg points out, "angels
	are powers which transcend the logic of our existence," did it follow that one is constrained to
	abandon logic in order to entertain angel For the sake of angels I was ready to subscribe to
	Coleridge's "willing suspension of disbelief." I was even ready to drlnk his "milk of Paradise."
	But I was troubled. Never a respecter ofauthority, per se, particularly when it was backed by the
	"salvific light of revelation," I nevertheless kept repeating to myself that I was pitting my per-
	sonal and necessarily circumscribed experience, logic, and belief (or nonbelief) against the
	experience, logic, and belief of some of the boldest and ~rofoundest minds of all times-minds
	that had reshaped the world's thinking and emancipated it (to a degree, at any rate) from the
	bondage ofsuperstition and error. Still, I was averse to associatiilg myselfwith opinions and creeds,
	no matter how hallowed by time or tradition, or by whomsoever held, that were plainly repugnant
	to colnnlon sense. A professed belief in angels would, inevitably, involve me in a belief in the
	supernatural, and that was the golden snare I did not wish to be caught in. Without committing
	inyself religiously I could conceive of the possibility of there being, in dimensions and worlds
	other than our own, powers and intelligeilces outside our present apprehension, and in this
	seilse angels are not to be ruled out as a part of reality-always remembering that we create what
	we believe. Indeed, I ail1 prepared to say that if enough of us believe in angels, then angels exist.
	In the course of much reading in patristic lore I came upon a saying by St. Augustine. It is
	taken froill his Eight Questions ("de diversis questionibus octoginta tribus"). I wrote down the
	saying on a piece of paper and carried it around with me for a long time, not as something I
	concurred in, but as a challenge. This is what Augustine said: "Every visible thing in this
	world is put under the charge of an angel." Genesis Rabba, 10, puts it somewhat differently:
	"There's not a stalk on earth that has not its [protecting or guardian] angel in heaven."
	Here and there, wherever it suited his thesis or purpose, St. Paul found angels wicked (as in
	Ephesians 6, etc.). In Colossians 2:17 he warns us not to be seduced by any religion of angels.
	Furthermore, God himself, it appears, "put no trust in his servants
	his angels he charged with
	folly" (Job 4:18). There was the further injunctioii in Hebrews 13, "Be not carried about
	with divers and strange doctrines." Sound advice ! And I was fain to say to Paul, as Agrippa the
	king said to him (in Acts 26: 38), "Almost thou persuadest me to be a Christian." But whose
	. . .
	
\begin{quote}
	{\small 	4. Walter Nigg's article "Stay you Angels, Stay with Me." Harper's Bazaar. December 1%2. The phrase
	derives from Joha1111
	Sebastian Bach's "Cantata for Michaelmas Day."}
\end{quote}
	
	
	
	
	INTRODUCTION
	[xiii]
	strange doctrines did Paul have in mind-Moses'? Isaiah's? Koheleth's? Peter's? St. James'?
	And if it is Paul who thus exhorts us in Hebrews (a book once reputedly his), one might ask: is
	Paul a trustworthy counselor and guide-a man who, as he himself admits, was "all things to all
	men," and who honored and repudiated angels in almost the same breath? One thing I soon
	realized: in the realm of the unknowable and invisible, in matters where a questioiler is finally
	reduced to taking things on faith, one can be sure of nothing, prove nothing, and convince
	nobody. But more of this anoil.
	One of the problems I ran into, in the early days of my investigations, was how to hack my
	way through the maze of changes in nomenclature and orthography that angels passed through
	in the course of their being translated from one language into another, or copied out by scribes
	from one manuscript to another, or by virtue of the natural deterioration that occurs with any
	body of writing undergoing repeated transcriptions and metathesis. For example: Uriel,
	presider over Tartarus" and "regent of the sun," shows up variously as Sariel, Nuriel, Uryan,
	Jehoel, Owreel, Oroiael, Phanuel, Eremiel, Ramiel, Jeremiel, Jacob-Isra'el. Derivations and/or
	variations of Haniel, chief of principalities and "the tallest angel in Heaven," may be set down in
	mathematical equations, to wit: Haniel = Anael = Anfiel = Aniyel = Anafiel = Onoel =
	Ariel = Simiel. The celestial gabbai, keeper of the treasuries of Heaven, Vretil, turns out to be
	the same as, or can be equated with, or is an aphetic form of, Gabriel, Radueriel, Pravuil, Seferiel,
	Vrevoil. In Arabic lore, Gabriel is Jibril, Jabriel, Abrael, or Abru-el, etc. In ancient Persian lore he
	was Sorush and Revan-bakhsh and "the crowned Bahman," mightiest of all angels. To the Ethio-
	pians he is Gadreel.
	Michael had a mystery name: Sabbathiel. He passed also for the Shekinah, the Prince of
	Light, the Logos, Metatron, the angel of the Lord, and as St. Peter (for Michael, also, like the
	prince of apostles, holds-or held-the keys of the kingdom of Heaven). In addition, as the
	earliest recorded slayer of the Dragon, Michael may be considered the prototype of the redoubt-
	able St. George. To the ancient Persians he was known as Beshter, sustaiher of mankind.
	Raphael, "christened" Labbiel when God first formed him, is interchangeable with Apha-
	rope, Raguel, Ramiel, Azrael, Raffarel, etc. And, to make matters more complicated, our healing
	angel operated under a pseudonym, Azariah (as in The Book of Tobit). The Zohar equates
	Raphael with a king of the underworld, Bael.
	The archangel Raziel, "chief of the Supreme Mysteries," and "author" of the famous S+r
	Raziel (Book of the Angel Raziel), answers to Akraziel, Saraqael, Suriel, Galisur, N'Zuriel, and
	Uriel. The seraph Semyaza may be sunlnloned up by the pronouncelllent of any of a string of
	variations oil his name-Samiaza, Shenlhazai, Amezyarak, Azael, Azaziel, Uzza.
	Metatron, the "lesser YHWH" (i.e., the lesser God) and twin brother of Smdalphon, also
	had a mystery name, Bizbul. But Metatron had more than loo other names (see Appendix)
	and in magical rites he could be invoked by any of them.
	The leopard-bodied Camael (alias Shemuel, Simiel, Quemuel, Kemuel), while serving in
	Hell as a Count Palatine and ruler of the wicked planet Mars, served at the same time in Heaven
	as an archangel of the divine presence. It was Canlael (Kemuel) who accompanied God with a
	
	
	
	
	[xiv]
	INTRODUCTION
	troop of 12,000 spirits at the promulgation of the Holy Law. This is vouched for in legend.'
	According to another legend,6 Canlael was destroyed by Moses when he tried to hinder the
	Lawgiver from receiving the Torah at the hand of God.
	Satan paraded under, or hid behind, a bewildering array of forms and incarnations. The
	prince of the power of the air," as Paul picturesquely dubs him, is our best example of a quick-
	change artist in guises and appellatives. In ~ o r o a ~ t r i a theosophy
	n
	he is Ahriman, enemy of man
	and God, a kind of ur-Satan (since Ahriman antedates by 1,000 years the Judaeo-Christian image
	of a prince regent of evil). In Leviticus, he is Azazel, the "goat of the sin offering." In Isaiah he is
	Lucifer (or, rather, mistakenly identified as Lucifer). In Matthew, Mark, atid Luke he is Beelze-
	bub, "lord of flies." In Revelation he is "that dragon and old serpent, the Devil." He is Mastema
	andlor Beliar in The Book ofjubilees and The Book ofAdam and Eve. He is Sammael in Baruch III,
	The Chaldean Paraphrase ofJonathan, and The Martyrdom of Isaiah. In Enoch he is Satanail and'
	Salamiel. In The Apocalypse ofAbraham and The Zohar he is Duma as well as Azaze1:In Falasha
	lore he is Suriel, angel of death. And he is Beliar or Belie1 in The Testament of the Twelve Patri-
	archs, The Zadokite Fragments (where Mastema also figures as an alternate to Beliar), and The
	Sibylline Oracles. In the Koran he is Iblis or Eblis or Haris. And in Jewish tradition he is Yetzer-
	hara, the personified evil inclination in man. To Shakespeare (I Henry IV) he is the "Lordly
	monarch of the north"; to Milton (Paradise Regained IV, 604) he is the "Thief of Paradise";
	to Bunyan (Holy War) he is Diabolus.' But whatever h s guise, the once familiar peripatetic
	of Heaven is no longer to be found there, as guest or resident; nor is it likely that the black
	divinity of his feet will ever again be sighted on Fhe crystal battlements-unless he is forgiven and
	reinvested with his former rank and glory, an eventuality the Church forbids its followers to
	entertain as possible or desirable, since Satan and his angels have been cursed by the Savior
	Himself "into everlasting
	fire" (Matthew 25 :41).
	-
	Hell itself, one adduces from Enoch II, Testanrent oflevi, and other apocryphal and pseudepi-
	graphic works, is not located where one would ordinarily suppose it to be, i.e., in the under-
	world, but in the "northern regions of the 3rd Heaven," while Evil in its various aspects is
	lodged in the h d as well as the 3rd and 5th Heavens.' The first 3 Heavens, according to the
	Baruch Apocalypse (Baruch III), are " f d of evil-looking monsters." In the 2nd Heaven the fallen
	angels (the amorous ones, those that coupled with the daughters of men) are imprisoned and
	daily flogged. In the 5th the dread Watchers dwell, those eternally silent Grigori "who, with their
	~
	Paul was caught up in the 3rd Heaven, he en-
	prince Salamiel, had rejected the L ~ r d . " When
	t t
	
\begin{quote}
	{\small 	5. R f . Moses Schwab, Vocabulaite de I'dngdlologie. According to Rabbi Abdimi, no less than 22,000 ministering
	angels descended on Mt. Sinai on this historic occasion (see Midrash Tehillim on Psalm 68).
	6. Louis Ginzberg, The Legends oftheJews 111,110.
	7. A recent writer, Jean Lhermitte (True and False Possession, 1%3), holds that "The Prince of Darkness no
	longer appears as a personage . . but disguises himselfwillingly, even preferably, under the appearance of corporate
	personalities or institutions."
	8. C. E. S. Wood, the American poet, in his Heavenly Discourx, gives Satan'sP.0. address as Washington, D.C.
	That was back in 1927. His Satanic Majesty may have moved since then.
	9. This must have been in the "north of the 5th Heaven, for elsewhere in the same Heaven, whither Zephaniah
	claims a Spirit conveyed him, the Old Testament Prophet "beheld angels that arc called Lords, and each had a crown
	upon his head as well as a throne shining seven times brighter than the ~un"--~uoted
	by Clement of Alexandria from
	the lost Apocalypse ofztphaniah.}
\end{quote}
	
	
	
	
	
	INTRODUCTION
	[xv]
	countered there "angels of evil, terrible and without pity carrying savage weapons."1� In a
	word, at least 3 Heavens, or regions of at least 3 Heavens, were thc abode of the eternally damned.
	Now, to fmd Hell in Heaven should not have surprised this writer, or anyone with a
	smattering of Greek mythology, for the paradisiacal Elysian Fields, "residence of the shades of
	the Blessed," are in the immediate vicinity of Hades. A rabbinic commentary (Midrash Tannaim)
	vouches for the fact that Hell and Paradise are "side by side." This is close to what one finds in a
	commentary on Psalm 90 (Midrash Tehillim) where it is stated that there were seven things which
	preceded the creation of the world, and that among the seven things were Paradise and Hell,
	and that "Paradise was on the right side of God, Hcll on the left." In a commentary on Ecclesi-
	astes (Yalkut Koheleth) we learn that the two realnls are actually only "a hand-breadth apart."
	This carefully calibrated survey is attributed to the Hebrew sage, Rab Chanina (Kahana), of the
	late 3rd century c.E.' '
	How incongruous, indeed how anoinalous it was to plant Hell in Heaven must have occurred
	finally to the Great Architect Himself for, one day, without f w or fmfare, the entire apparatus
	of evil-the arseilals of punishment, the chief Scourgers, the apostate angels, the horned or
	aureoled spirits of wrath, destruction, confusion, and vengeance-was moved from the upper
	to the lower world, where (if it is not too presumptuous to say so) all such paraphernalia andper-
	sonnel should have been installed in the first place.
	The noted scholar R. H. Charles, in his introductioil to Morfill's translation of Enoch II,
	observes in a footnote that "the old idea of wickedness in Heaven was subsequently banished
	from Christian and Jewish thought." True, and none too soon. For what assurance otherwise
	would thc faithful have been given that, oil arrival in Heaven, they would not be lodged in one
	of the enclaves of Hell?
	Perhaps the best-or worst--example of the confusion to be found in noncanonical as well
	
	\begin{quote}
		{\small 10. The fact that in Paul's day there still were angels of evil in Heaven "carrying sava e weapons" would lead
	one to suppose that the fighting on high did not end with Satan's rout,and that Michael a his hosts won a Pyrrhic
	victory, or at best a truce.
	11. In this connection, the expression "Abraham's bosom" (Luke 16). interpreted as denoting "the repose of the
	happy in death," may be cited here. The Apostles' Creed a f f i thatJesus descended to Hell after the Crucifixion.
	purportedly to liberate the "saints in chins' (the unbaptized patriarchs, Abraham among them) in order to transport
	them to Paradise. The parable in Luke presupposes that Abraham is already there; and the fact that the rich ma0 in
	Hades (Dives) is able to converse with Abraham across the "great chasm" s a t s that the chasm was not very wide,
	and that, hence, Heaven and Hell were very dose to each other, at least m x n speaking distance. Purgatory, it will
	be noted, is not mentioned. The explanation is simple: it did not exist-not, anyway, until 604 C.E. Gregory the
	Great invented it. Perhaps invention is too strong a term. Gregory very likely ap ropriated the notion of an "upper
	Gehenna" from the ancient Jews, or from the empyrosis of the G m k stoics, or ' ! t om the twelve cycles of purgation
	of Zoroaster. Be that as it may, Purgatory was made official-it was "legislated into existence"-by decreer at the
	Council of Lyons in 1274, at Florence in 1439, and again in the 1540's at the Council of Trent, and is today pan of
	the religious belief of all or most Christians, except members of the Church of England which, in 1562, condemned
	Purgatory as "a fond thing vainly invented and grounded upon no warranty of Scripture, but rather repugnant to
	the Word of God." We know of no angels, fair or foul, inhabitin or frquentin the lace. According to Origen,
	it is reserved for souls waiting to be purged of the "lighter materi s" of their sins ? 'so that
	P they may enter the kin
	dom of Heaven undefiled." The duration of souls in Purgatory, an indefinable time, may be cut down by indu
	gences, prayers, and paid masses. Jews have their Yiskor, which is a prayer for the repose of the dead and is recited
	on Yom Kippur, Sukkot, Passover, and Shavuoth. Where these Jewish dead arc repodn is not clear. The Moslems
	have their A1 Aaraaf, a region for "those who a n [found] neither good nor bad, such as in nts, lunatics, and idiotsm--
	Readtr's Encyclopedia, "Ara f."
	}
	\end{quote}
	
	
	[xvi]
	INTRODUCTION
	as canonical lore is the case of Satan. The Old Testament speaks of an adversary, ha-satan. It is a
	term that stood for an office; it did not denote the name of an angel. To the Jews of Biblical
	times the adversary was neither evil nor fallen (the Old Testament knows nothing of fallen angels),
	but a servant of God in good standing, a great angel, perhaps the greatest. However, he is no-
	where named. In Job he presents himself before the Lord in the company of other unnamed
	"sons of God." There is no question of his being evil or apostate.'' The one instance where
	ha-satan is given as satan without the definite article (I Chronicles 21), is now generally conceded
	to be a scribal oversight. In a word, the Old Testament did not name its angels, except in Daniel,
	a late, postexilic book. There oilly two angels are named: Michael and Gabriel (names, by
	the way, that owe their origin to Babylonian-Chaldean sources). In the New Testament, on
	the contrary, Satan is unequivocably a person, so named. Here he is no longer the obedient ser-
	vant of God, the "prime in splendour," but the castout opponent and enemy of God, the Prince
	of Evil, the Devil incarnate.
	The transformation of ha-satan in the Old Testament into Satan in the New, and the con-
	flicting notions that arose as a consequence, are pointed up by Bernard J. Bamberger in his
	Fallen Angels: "The classic expositions of the Jewish faith have in~plicitly or explicitly rejected the
	belief in rebel angels and in a Devil who is God's enemy. . . The Hebrew Bible itself, correctly
	interpreted, leaves no room for a belief hi a world of evil powers arrayed against the goodness
	of God. . Historical Christianity, on the other hand, has consistently affirmed the continuing
	conflict between God and Satan." This continuing conflict between God and Satan, one might
	add, is little more than a recrudescence, with modifications, of the dualistic system that Christi-
	anity (along with Jewish sectarians of the post-Biblical era) inherited from Zoroastrianism.
	Equally difficult to deal with was the question whether (and how many) other spirits in the
	celestial hierarchy were good or evil, fallen or still upright, dwellers of Heaven or Hell. This was
	a specially baffling problem and left me wandering about in a perpetual cloud of unknowing.
	A case in point: In Enoch I, 6, Remiel is styled "one of the leaders of the rebel angels." Farther
	along in the same book, Chapter 18, Rerniel is metamorphosed into "one of the seven holy ones
	whom God set over those who rise." In Revelation 9, Abaddon/Apollyon is the "angel of the
	bottoniless pit," suggesting an evil spirit in the sense of a destroyer; but in Revelation 20,
	Abaddon/Apollyon is manifestly good and holy, for here he is said to have "laid hold on the
	dragon, that old serpent, who is the devil and Satan, and bound hini a thousand years" (in The
	Greater Key of Solomon Abaddon is "a name for God that Moses invoked to bring down the
	t " the Dutch Shakespeare (1587-1678), tells us in his Lucifer
	blighting rain over ~ ~ ~ !). p Vondel,
	that Apollyon was known in Heaven, before he joined Satan, as the hierarch "of the snowy
	wings." To Bunyan in Pilgrim's Progress Apollyon is an out-and-out devil, the devil, just as he is
	.
	. .
	
	\begin{quote}
		{\small 12. The hasidic rabbi Yaakov Yitzhak of Pzysha, known as the holy Yehudi (d. 1814), makes this clear when he
	declared that "the virtue of angels is that they cannot deteriorate." See Martin Buber, Tales ofthe Hasidirr~, Lclter
	Masters, p. 231. The fact that the adversary challenges God or questions Him does not, ipso k t o , make the adversary
	evil or an opponent of God-just as, when Abraham and Job "put God to the uestion, 1, they were not, on that
	account, regarded as evil men, or even as presumptuous men. See Harry M. Orlin y's Ancient Imel, p. 30.
	}
	\end{quote}
	
	
	
	
	
	INTRODUCTION
	I
	I
	[xvii]
	in secular writings generally.13 Other examples, to cite a handful: Ariel, "earth's great Lord"
	and an aide to Raphael in the curing of disease, is at the same time a rebel angel hl charge of pun-
	ishments in the lower world. Kakabel, a high holy prince who exercises doillinion over the con-
	stellations, is in Enoch one of the apostates. The angel Usiel, Gabriel's lieuteilaiit in the fighting
	on high, is designated a coillpanion of the lustful luminaries who coupled with mortal women;
	in Zoharic cabala he is the cortex (averse demon) of Gog Sheklah, "disturber of all things."
	Anloilg the rabbis the opiilion is divided with regard to the 90,000 angels of destruction. Are
	they in the service of God or the Devil? Pirke Rabbi Eliezer inclines to the latter view. In the
	Pirke they are called "angels of Satan."
	It is well to bear in mind that all angels, whatever thcir state of grace-indeed, no matter
	how christologically corrupt and defiant-are under God, even when, to all intents and purposes,
	they are performing under the direct orders of the Devil. Evil itself is an instrun~entality of the
	Creator, who uses evil for His own divine, if unsearchable, ends. At least, such may be gathered
	from Isaiah 45 :7; it is also Church doctrine, as is the doctrine that angels, like human beings, were
	created with free will, but that they surrendered their free will the-moment they were formed.
	At that moment, we are told, they were given (and had to make) the choice between turning
	toward or away from God, and that it was an irrevocable choice. Those ailgels that turned
	toward God gained the beatific vision, and so becaine fixed eternally in good; those that turned
	away from God became fixed eternally in evil. These latter are the demons, they are rlot the
	fallen angels (an entirely different breed of recusants which hatched out subsequently, on Satan's
	defection). Man, however, continues to enjoy free will. He can still choose between good and
	evil. This may or may not work out to his advantage; more often than not it has proved his
	undoing. The best that man can hope for, apparently, is that when he is weighed in the balance
	(by the "angels of final reckoning"), he is not found wanting.14
	Angels perform a multiplicity of duties and tasks. Preeminently they serve God. They do
	this by the ceaseless chanting of glorias as they circle round the high holy Throne. They also
	carry out missions from God to man. But inany serve man directly as guardians, counselors,
	guides, judges, interpreters, cooks, comforters, dragomen, matchmakers, and gravediggers.
	They are responsive to invocations when such invocations are properly formulated and the
	conditions are propitious. In occult lore angels are conjured up not only to help an invocant
	strengthen his faith, heal his afflictions, find lost articles, increase his worldly goods, and procure
	offspring, but also to circuinvelit and destroy an enemy. There are instances where an angel or
	
	\begin{quote}
		{\small 13. In Jewish lore, abaddon is a place-sheol, the pit, or the grave; nowherc is it thc naille of an angel or demon.
	The term is personified for the first tiine in Revelation and a pears as Abaddon (cap A). St. John makes Abaddon
	s nonymous with Apollyon and declares it to be the Greek orm of the sallle angel. The Confraternity edition of
	t e New Testament adds here (Apocalypse 9: 11): "in Latin he has the name Exterininans." On the other hand, The
	Magus, which offers a number of portraits of the archfiends in color, splits Abaddon and Apollyon into two separate
	and distinct "vessels of iniquity," showing Abaddon with tawny hair and RoilIan nose, Apollyon with russet beard
	and hooked beak.
	14. According to Abbot Anxar Vonier in The Teaching 4 t h e Catholic Church (1964). angels still enjoy free will.
	This seems to be another or new interpretation of Catholic doctrine on the subject.
	}
	\end{quote}
	
	
	
	
	
	[xviii]
	INTRODUCTION
	troop of angels turned the tide of battle, abated storms, conveyed saints to Heaven, brought down
	plagues, fed hermits, helped plowmen, converted heathens. An angel multiplied the seed of
	Hagar, protected Lot, caused the destruction of Sodom, hardened Pharaoh's heart, rescued Daniel
	from the lions' den, and Peter from prison. To come closer to our own times: it will be recalled
	that when Spinoza was "execrated, cursed, and cast out" from his community in Amsterdam
	for holding anlong other "heretical views" that "angels were an hallucination," the edict of
	excommunication against him was drawn up by the rabbis "with the judgment of the angels."
	The might of angels, as made known to us in Targunl and Talmud, is easily a match for
	the might of the pagan gods and heroes. Michael overthrew nloungins. Gabriel bore Abraham
	on his back to Babylon, whither an unnamed angel later conveyed the prophet Habbakuk (by
	the hair) from Judea, to feed Daniel pottage." Jewish legend tills us that, during the siege of the
	Holy City by Nebuchadnezzar, "the prince of the world" (Metatron? Michael? or perchance
	Satan?) lifted Jerusalem "high in the air" but that God thrust it down again.16 We know from
	Revelation that seven angels of the wrath of God smote a "third part of the stars." The mighty
	Rabdos is able to stop the planets in their courses. The Talmudic angel Ben Nez prevents the
	earth froin being consumed by holding back the South Wind with his pinions. Morael has the
	power of inaking everythmg in the visible world invisible. The Atlantean Splenditenes sup-
	ports the globe on his back. Ataphiel (Barattiel), hierarch of Merkabah lore, keeps Heaven from
	tuillbliilg down by balancing it on three fingers. The Pillared Angel (mentioned in Revelation)
	supports the sky on the palin of his right hand. Chayyiel, the divine angel-beast, can-if he is so
	minded-swallow the whole world in a single gulp. When Hadraniel proclaims the will of God,
	CL
	his voice penetrates through zoo,ooo firn~amei~ts."
	It was Hadraniel who struck Moses "dumb
	with awe" when the Lawgiver caught sight of the dread luminary in the 2nd Heaven. As late as
	the 17th century, the German astronomer Kepler figured out (and somehow managed to fit
	into his celebrated law of celestial mechanics) that the planets are "pushed around by angels."
	A briefword about the number of angels abroad in the world. Since the quantity, according
	to Church doctrine, was fixed at Creation, the aggregate must be fairly constant. An exact figure
	-301,655,722-was
	arrived at by 14th-century cabalists, who employed the device of "calcula-
	ting words into numbers and numbers into words." This is a very modest figure if we regard
	stars as angels (just as the Apocalyptics did: John in Revelation, Clement of Alexandria in
	Stromata VI, etc.) and include them in the total." Thomas Heywood in his Hierarchy cautions
	us metrically: "Of the Angels, th'exact number who/Shall undertake .to tell, he shall growl
	
\begin{quote}
		{\small 15. See apocryphal additions to Daniel 5:86.
	16. In 1291-1294 c.E., angels moved the house of the Virgin Mary from Nazareth to Dalmatia. thence to
	various parts of Italy, finally depositing it in the village of Loretto. The miraculous haulage is the subject of a canvas
	(now in the Morgan Library. New York), by the 15th-16thcentury artist Satume di Gatti.
	17. Rabbi Jochanan (Talmud Hagiga 14a) reminds us that, far fro111 having ceased being formed at Creation,
	angels are born "with every utterance that goes forth from the mouth of the Holy One, blcssed be He." The Jewish
	notion of a continuing act of Creation (as opposed to the rota sirrrul doctrine of the early Church) is traditional in
	Talmud, and embraces not only angels but all things formed in the first six days. This is clear from a hymn found in
	Greater Hechaloth 4: 2, where God is praised for not ceasing to create "new stars and constellations and zodiacal signs
	that flow and issue from the light of His holy garment."
	}
\end{quote}
	
	
	
	
	INTRODUCTION
	[xix]
	From Ignorance to Error; yet we may/Conjecture." Albertus Magnus conjectured, and put
	"each choir at 6,666 legions, and each legion at 6,666 angels." But demons are winged horses
	of another color. Unlike the angels, these apes of God are capable of reproducing their kind.
	What is more, as Origen alerts us, "they multiply like flies." So today there must be a truly
	staggering horde of them. The problem of populatioil explosion here is clearly something to
	worry about."'
	As for the vernacular employed by angels, the odds favor Hebrew. In The Book ofjubilees
	and in Targum Yerushalmi, we learn that the language God used at Creation and in the Garden
	of Eden was Hebrew. Even the serpent spoke Hebrew, according to Midrash Lekah Genesis 31 : 1.
	So, inferentially, angels also spoke it, or speak it. The Apocalypse of Paul puts it precisely:
	"Hebrew, the speech of God and the angels." Indeed, in rabbinic lore, and in sundry secular
	writings, Hebrew is said to have been the language of all mankind up to the "confusion of
	tongues," an event that occurred at the building of the Tower of Babel in 2247 B.C.E. (as conl-
	puted by Archbishop Ussher, noted 17th-century Irish the~logian).'~
	That the Torah was originally conceived and set down in Hebrew is a widely postulated
	view among Jews, though disputed by Philo (who thought the language was Chaldean Aramaic)
	On the
	and by Muslims generally (who claim it was Arabic). St. Basil thought it was Syria~.~'
	whole it is safe to say that the linguafranca of angels--of all spirits, in fact-is Hebrew. Some
	exegetes hold that angels, being monolingual, speak the holy tongue exclusively, not even
	understanding the closely related Aramaic (as specifically stated in The Zohar I, 92); other
	authorities contend differently. They point out that Gabriel, Metatron, and Zagzagel each had
	a knowledge of seventy lang~ages.~'
	hl recent times, Sandalphon was overheard conversiilg in
	Yiddish, the eavesdropper being the storyteller Isaac Bashevis Singer. Furthermore, we have it
	on the word of the Swedish mystic Swedenborg that angels not only speak Hebrew, they also
	write it. In his Heaven and Its Wonders and Hell, he avers that "a little paper was sent to me from
	Heaven on which a few words were written in Hebrew." Thls remarkable document, so far as
	is known, was never produced for public scrutiny, nor has it ever turned up among Sweden-
	borg's effects.
	Are angels inlmortal? In the opinion of most scholars, yes. But are angels eternal? No.
	God alone is eternal.22 Still, the life span of angels is a fairly long one, starting from the moment
	they were "willed into being" to the last crack of doom. But a number of angels have, mean-
	
\begin{quote}
		{\small 18. Luther's followers, in a work entitled Theatrunc Diabolorum, not satisfied with the then-current estimates of
	devils, raised the figure to 2.5. billion, later to 10.000 billion. A reassuring thought, provided by Hagiga 16a, is that
	while "demons beget and increase like men, like men they die."
	19. At the Exodus and in the Wilderness, God also spoke Hamitic. He did this, it is said, in order to make Hiin-
	self understood by the Egyptian Moses and by Hamitic-qxaking Jews who made up the greater bulk of Moses'
	followers.
	20. See The Book ofAdam andEve. p. 245.
	21. Talmud Sotah, fol. 36, narrates that Gabriel taught Joseph seventy languages overnight. The ailgel Kirtabus
	(in Tyana's Nuctemeron) is described as a "genius of languages."
	22. John of Damascus qualifies this by saying in his Ewposition ofthe Orthodox Faith: "God alone is eternal, or
	rather. He is above the Eternal; for He, the Creator oftimes, is not under the dominion ofTime, but above Time."
	}
\end{quote}
	
	
	
	[xx]
	INTRODUCTION
	while, been snuffed out.23 Thus God put an end to Rahab for refusing, as commanded,
	to divide the upper and lower watemZ4 God burned the angels of peace and truth, along with
	the hosts under them, as well as an entire legion of administering angels (Yalkut Shimoni), for
	objecting to the creation of man-a project the Creator had His heart particularly set on and was
	determined to carry through, although later He repented of the venture, as we learn from
	Genesis 6:6. God also annihilated a whole "globe of angels," the Song-Uttering Choristers, for
	failing to chant the Trisagion at the appointed hour. And there is the case of a mortal doing
	away with an immortal: Moses, who in fact did away with two of them-Kemuel (already
	mentioned) and Hemah. This Hemah was the angel of fury "forged at the beginning of the
	world out of chains of black and red fire." Legend has it that, after swallowing the Lawgiver
	up to the ankles, Hemah had to disgorge him at the timely intervention of the Lord. Moses then
	turned around and slew the vile fiend.
	While there are numerous instances of angels turning into demons, as exemplified in the
	fall ofone-thrd of the Heavenly hosts (Revelation 12), instances of mortals transformed into angels
	(named angels) are rare.25 Four instances have come to light, three deriving from passages in
	Genesis and I1 Kings. The first relates to the patriarch Enoch, who was apotheosized into the
	god-angel Metatron. The second relates to the patriarch Jacob, who became Uriel, then Isra'el,
	"archangel of the power of the Lord" and chief tribune among the sons of God.26 The third
	relates to the prophet Elijah, who drove to Heaven in a fiery chariot and, on arrival, was trans-
	formed into the angel Sandalphon.27 The fourth instance, vouched for in The Douce Apocalypse,
	.~~
	instance is the transforming
	is that of St. Francis, who evolved into the angel R h a n ~ i e l Another
	
\begin{quote}
		{\small 23. The noted 12th-century Jewish poet and theologian, Judah ha-Levi (1085-1140) in his work called The
	Book of Kuzari (IV), taught that there were two classes or species of angels. He wrote: "As for the angels, some are
	created for the time being, out of the subtle elements of matter (as air or fire). Some are eternal (i.e., existing from
	everlasting to everlasting), and perhaps they are the spiritual intelligences of which the philosophers speak." He goes
	on to say: "It is doubtful whether the angels seen by Isaiah, Ezekiel, and Daniel were of the class of those created
	for the time be in.
	". or of the class of sviritual essences which are eternal." What were thev then? one might ask.
	Saadia B. Joseph was of the opinion that they were visions seen during prophetic ecstasy rather than outwaFd reali-
	ties. In the view of St. John ofDalrlascus (700?-754?), Orthodox Faith, angels are immortal, but "only by grace, not
	by nature."
	24. This "angel of insolence and pride" had two lives. He was deprived of the first for the reason given above.
	at the Exodus. Here he is drowned bv God
	Two thousand vears later. resuscitated but still obdurate. he reaooears
	1 1
	for espousing the cause of the Egyptians, which, as that nation's tutelary angel, he was honor bound to do.
	25. Origen's belief in a "final restitution," when God would forgive all his sinning creatures, even the most
	damned, opened the door to a return of Satan to his archangelic perch in the Heavenly purlieus. Because of this
	heretical belief Origen, it is said, was never canonized.
	26. Prayer ofloseph.
	27. Elijah-Sandalphon became the celestial psychopomp "whose duty it was," says Pirke R. Eliezer, "to stand
	at the crossways ofparadise and guide the pious to their appointed places."
	28. According to Jewish tradition, all patriarchs, along with those who led exceptionally virtuous lives,
	attained angelic rank when they got to Heaven. This, however, has been disputed: "the belief that the souls of the
	ri hteous after death become angels has never been part of Jewish thought" (Universal Jewish Encyclopedia I, 314).
	T i a t it was at one time part a f patristic thinking can be deduced from Theodotus (Excerpts) to the effect that "those
	who are changed from men to angels are instructed for a thousand years by the angels, after they are brought to
	perfection" and that then "those who have been taught are translated to archangelic authority."
	}
\end{quote}
	
	
	
	
	INTRODUCTION
	[xxi]
	of Anne, the virgin's mother, into the angel Anas (q.v.). Mention might also be made here of
	three Biblical psalmists-Asaph, Hcnlan, and Jeduthun-who showed up in Heaven, with their
	earthly names and occupations unchanged, as celestial choirmasters.
	Regarding the sex or gender of angels, I was often hard put to arrive at any conclusion in
	the matter, even with the help of scholars. True, angels are pure spirits and so should be presumed
	to be bodiless and, hence, sexless.29 But the authors of our sacred texts were not logicians or
	men of science; in the main, they were prophets, lawgivers, chroniclers, poets. They did not
	know how to represent invisible spirits other than by giving them visible, or tangible, embodi-
	ment: accordingly, they pictured angels in their own image (i.e., in the guise of men), acting
	and talking and going about their business-the Lord's business-the way men do.30 Angels in
	Scripture, as a consequence, were conceived of as male.31 However, it was not long before the
	female of the species began putting in an appearance. In early rabbinic as well as in occult lore,
	there are quite a number of them: the Shekinah, for one. She was the "bride ofGod," the divine
	intvohnurg in man, who dwelt with lawfully wedded couples and blessed their conjugal union.
	There was Pistis Sophia ("Faith/Wisdom"), a high-ranking gnostic aeon, said to be the "pro-
	creator of the superior angels." There was ~ a r b e l o consort
	,
	of Cosn~ocrator, a great archon,
	"perfect in glory and next in rank to the Father-of-All." There was Bat Qol, the "heavenly
	voice" or "daughter of the voice" ofJewish tradition, a prophetess symbolized as a dove, who
	gave warnings and counsel when the days of prophecy were over. Another female power that
	comes to mind is the gnostic Drop or Derdekea. According to the Berlin Codex, Drop used to
	descend to earth on critical occasions "for the salvation of mankind." And there were the six
	left-side enlanations of God, created to counterbalance the ten male emanations that issued from
	God's right side.32 And finally there was the vixen Eisheth Zenunim, angel of prostitution and
	mate of Sammael. In Hebrew, eisheth zenunim meails "woman of whoredom" and the epithet
	applied with equal force to three other wives of Sammael: Lilith, Naamah, Agrat bat Mahlah.
	-
	-
	
\begin{quote}
		{\small 29. In theology there are three classifications of spirit: (I) God, W h o is divine spirit; (2) angels and demons,
	who are pure spirits; and (3) man, who is impure spirit.
	30. In The Zohar (Vayera 101a) we read: "When Abraham was still suffering from the effects of the circum-
	cision, the Holy One sent him 3 angels, in visible shape, to enquire of his well-being." And the text goes on to say:
	"You may perhaps wonder how angels can ever be visible, since it is written, 'Who makes his angels spirits' (Psalms
	104:4). Abraham, however, assuredly did see them, as they descended to earth in the form of men. And, indeed,
	whenever the celestial spirits descend to earth, they clothe themselves in corporeal elements and appear to men in
	human shape." But it is difficult to reconcile the foregoing with the statement in The Book oflubilees (15:27) that
	"all the angels of the presence and all the angels of sanctification" were already circumcised when they were created.
	O n the issue of the materiality of angels, authorities have been divided. Those who believe that angels are composed
	of matter and form include Alexander of Hales, Bernard of Clairvaux, St. Bonaventura, Origen. Those who hold,
	to the contrary, that angels are incorporeal include Dionysius the Areopagite, John of Rochelle, Moses Maimonides.
	Maximus the Confessor, and William of Auvergne.
	31. The Koran 53:27: "Those who disbelieve in the Hereafter [are those who] name the angels with thenames
	of females."
	32. In the texts of the early commentators, Moses of Burgos and Isaac Ben R. Jacob ha-Cohen, as in the supple-
	ment to The Zohar, there are also ten evil emanations (male), of which "only seven were permitted to endure."
	See Appendix.}
\end{quote}
	
	
	
	
	
	
	
	[xxii]
	INTRODUCTION
	This free-loving quartet constituted a kind of composite Jewish equivalent of the Sidonivl
	Astarte.
	Zoroastrianism, which was not averse to including females in its pantheon, had its Anahita,
	a lovely luminary characterized as "the immortal one, genius of fertilizing waters." Offsetting
	her was Mairya, evil harbinger of death, represented indiscriminately as male and female. She
	(or he) tempted Zoroaster with the kingdoms of the earth, just as, in Matthew 4, Satan tempted
	Jesus. Another angel of indeterminate scx was Apsu. In Babylonian-Chaldean mythology, Apsu
	was the "female angel of the abyss"; but, though female, she fathered the Babylonian gods and
	was at the same time the husband or wife of Tamat. She (or he) was slain finally by her (his)
	son Ea. A true tumtum !33 It seems, aho, according to Genesis Rabba and confirmed by Milton
	in Paradise Lost I, 423-424, that angels; at least some of them, were able to change their sex at
	will. The Zohar (Vayehi z3zb) phrases it this way: "Angels, who are God's messengers, turn
	themselves into different shapes, being sometimes female and sometimes male."
	To revert to the question as to whether angels have an existence outside Holy Writ, or
	apart from the beliefs and testimony of visionaries, fabulists, hermeneuts, ecstatics, etct Such a
	question has been a debatable one from almost the start, even before the down-to-earth Sad-
	ducees repudiated them and the apocalyptic Pharisees acknowledged and espoused them.
	Aristotle and Plato believed in angels (Aristotle called them intelligences). Socrates, who
	believed in nothing that could not be verified by (or was repugnant to) logic and experience,
	nevertheless had his dainion, an attendant spirit, whose voice warned the marketplace philosopher
	whenever he was about to make a wrong decision.34 Now, to invent an angel, a hierarchy, or
	an order in a hierarchy, required some illlagination but not too much ingenuity. It was sufficient
	merely to (1) scramble together letters of the Hebrew alphabet; (2) juxtapose such letters in
	anagrammatic, acronymic, or cryptogrammatic form; (3) tack on to any place, property,
	fulction, attribute, or quality the theophorous "el" or "irion." Thus Hod (meaning splendor,
	hke zohar) was transformed into the angel Hodiel. Gevurah (meaning strength) burgeoned into
	the angel Gevurael or Gevirion. Tiphereth (meaning beauty) provided the basis for the sefira
	Tipherethiel. The lords of the various hierarchic orders came into being in similar fashion,
	Cherubiel becoming the eponymous chief of the order of cherubim ; Seraphiel, the epollymous
	chief of the ordcr of seraphim; Hashmal, of the hashmallim, etc. Countless "paper angels" or
	"suffix angels," many of them unpronounceable and irieducible to intelligent listing, were thus
	fabricated; they passed, virtually unchallenged, into the religious and secular literature of the
	day, to be accredited after a while as valid. In some cases they were given canonical or deutero-
	canonical status. The practice preempted no one from begetting ex nihilo and ad injnitunz his
	
\begin{quote}
		{\small 33. Tumtum is a Talmudic term for any spirit whose sex could not be easily determined. See M. Jastrow,
	Dictionary ofthe Targumin. TalmudBabli and Yerusalmi, and the Midrashim Literature.
	34. In the Middle Ages, the most eminent scholars and divines ranged themselves on opposite sides of the
	uestion. And that is perhaps still true today; a belief in angels is part of the doctrine of three of the four major
	Ziths-Chrktian (mainly Catholics), Jewish (mainly orthodox), Mohammedan.}
\end{quote}
	
	
	
	
	
	
	
	INTROD UCTZON
	[xxiii]
	own breed of angels, and putting them into orbit.35 The unremittq industry of early cabalists
	in creating angels spilled over into the raiding of pagan pantheons, and transforming Persian,
	Babylonian, Greek, and Roman divinities into Jewish hierarchs. Thus the kerubim of the ancient
	Assyrians-those huge, forbidding stone images placed before temples and palaces--emerge
	in Genesis 3 as animate cherubim, guardian angels armed with flaming swords east of Eden
	and, later, in upper Paradise, as charioteers of God (after Ezekiel encountered them at the River
	Chebar). The Akkadian lord of Hell, the li6n-headed Nergal, was cooverted into the great, holy
	Nasargiel, and in this acceptable guise served Moses as cicerone when the Lawgiver visited the
	underworld. Hermes, the good daimon, inventor of the lyre and master of song in Greek
	n~ythology, became in ~ e w i i h lore the angel Herrnesiel and identified with David, "sweet singer
	of Israel." The rabbinic Ashrnedai derived from the zend Aeshmadeva. Etc., etc.
	The Church, let it be said to its credit, tried to call a halt to the traffic, although
	the Church
	-
	itself at one time recognized a considerable number of angels not in the calendar, and even per-
	mitted them to be venerated.16 Scripture, as we have seen, gives the names of no more than two
	or three angels. That there may well be seven named angels in Scripture is the subject of a paper
	by this compiler; it is a thesis on which, admittedly, no two theologians are likely to agree.
	In the "orthodox" count, fixed by the 6th-century pseudo-Dionysius (otherwise known as
	Dionysius the Areopagite)?' there are nine orders in the celestial hierarchy. But there are other
	"authoritative" lisaprovided by sundry Protestant writers that give seven, nine, twelve orders,
	including such rarely encountered ones as flames, warriors, entities, seats, hosts, lordships, etc.
	The Dionysian sequence of the orders, from seraphim to angels (a sequelice for which there is
	no Biblical warrant, and which Calvin summarily dismissed as "the vain babblings of idle men")
	has likewise been shuffled about, some sources ranking seraphm last (rather than 6rst), archangels
	second (rather than eighth), virtues seventh (rather than fourth or sixth), and so 011.'
	Miracles, feats of magic, heavenly visitations, and overshadowings are often ascribed to
	
\begin{quote}
		{\small 35. Isaac de Acco (13th-14th century), a disciple of Nahmanides, "laid claim to the performance of miracles
	by a transposition of Hebrew letters according to a system he pretended to have learned from the angels." See
	A. E. Waite, The Holy Kabbalah, p. 53.
	36. Certain early theologians like Eusebius (c. 263-c. 339) and ~heodoret (c. 393-c. 458) opposed the veneration
	of angels, and a Church council at Laodicea (343-381?) condemned Christians "who gave themselves up to a masked
	idolatry in honor of the angels." This, despite the fact that St. Ambrose (339?-397) exhorted the faithful, in his
	De Viduis, 9, to "pray to the angels, who are given to us as guardians." In the 8th century, at the 2nd Council of
	Nicaea (787), there was another change of heart, for the worship of angelic bein s was then formally approved.
	The practice, nevertheless, seems to have fallen into disuse. Today there is a tren in some ecclesiastical circles to
	revive it. The Dominican priest Pie-Raymond RCgamey, author of What Is an Angel? (1960), thinks that veneration
	of angels is not a bad thing. but warns against the "danger of such devotion becoming superficial."
	37. The time that Dionysius lived and wrote has never been satisfactorily determined. Originally his writings
	were attributed to one of the 'udges of the Greek meopagus court), whom Paul converted (Acts 17:34). But scholars,
	finding such dating untenab e, moved the time ahead to tL e 6th century. However, according to a French legend
	cited by A. B. Jameson (Legends ofthe Madonna), "Dionysius the Areopagite was present at the death of the Virgin
	Mary," which would place him back in the 1st century. The legend relates that "Dionysius stood around the bier
	beside the twelve apostles, the two great angels ofdcath (Michael anabriel), and a host oflamenting lesser angels."
	38. Cf. varying sequenas of the ninefold hierarchy offered b Augustine (City o j Cod), Gre ory the Great
	(Hornilia and Moralia), Isidore of Seville (B mologiarum), Bernard o Clairvaux (de consideratione), Eund Spnser
	(An Hymne ofHeavenly Beautie), Drummon of Hawthorndcn (Flowres ofSion), etc.
	}
\end{quote}
	
	
	
	
	
	
	
	[xxiv]
	INTRODUCTION
	different angels.39 Thus, the three "men" whom Abraham entertained unawares have been
	identified as God, Michael, and Gabriel; also, according to Philo, as the Logos, the Messiah, and
	God. In ~ a t t h e h the
	, news of Mary being found with child of the Holy Ghost is conveyed to
	her spouse Joseph by the "angel of the Lord"; in Luke it is Gabriel who does the announcing-
	not to Joseph but direct to Mary who, however, seems to know nothing - of the matter. The over-
	night destruction of the army of Sennacherib, numbering 185,000 men, ascribed in I1 Kings to
	the "angel of the Lord," has been laid to the prowess of Michael, Gabriel, Uriel, or Remiel. No
	one has yet, to the knowledge of this investigator, identified the specific "angel of the Lord"
	whom David saw "standing between the earth and the heaven, having a drawn sword in his
	hand stretched over Jerusalem" (I Chronicles 2:16). A good guess would be Michael, for that
	battle-ax of God, when he is not in Heaven assisting Zehanpuryu or Dokiel in the weighing in
	of souls, is busy on earth lopping off the heads of the unfaithful.*O
	In their hurried exodus from Egypt, and in their encounter with Pharaoh's horsemen at the
	and
	Red (Reed) Sea, the Hebrews were helped by "the angel of God, which went before
	behind thein . . in a pillar of fire and cloud" (Exodus 14). Here the identity of the angel of God
	poses no problem: he was Michael or Metatron, each the tutelary prince-guardian of Israel.
	However, Michael or Metatron did not fight alone: he had the aid of a swarm of "ministering
	angels who began hurling [at the pursuing or retreating Egyptians] arrows, great hailstones, fire,
	and brim~tone."~' Present also, it is reported, were hosts of "angels and seraphim, singing songs
	of praise to the Lord," which must have helped considerably in turning the tide of battle.
	On the enemy side, harrying the Hebrews, was the guardian angel of Egypt, once holy but now
	corrupt. It appears though that Egypt had more than one guardian angel-four in fact, and that
	they all showed up, armed to the teeth. Various sources identify them as Uzza, Rahab, Mastema.
	and Duma. The fate of Rahab we know: he was drowned, along with the Egyptian horsemen.
	Mastema and Duma went back t o Hell, where they had unfinished business to attend to. As for
	Uzza, some authorities say he was actually Semyaza, grandfather of Og, a leader of the fallen
	angels; and that since the Red Sea episode, and after his uilfortunatc affair with the maiden
	~shtahar (immortalized in song by Byron), he hangs head down betwccn Heavcn and earth in
	the neighborhood
	of the constellation Orion. Indeed, Graves and Patai in their Hebrew Myths
	-
	say that Selnyaza is merely the Hebrew forill for the Greek Orion.
	.
	. . .
	
\begin{quote}
		{\small 39. Miracles and magic were not always frowned upon by the Church, despite Jesus' exhortation against
	signs and wonders as a basis for belief(John 4:48). When Pico della Mirandola (1463-1494) declared that "no science
	yields greater proof of angels, purgatory, hellfire, and the divinity of Christ than magic and the Kabbalah," Pope
	Sixtus IV "was delighted and had the Kabbalah translated into Latin for the use of students of divinity" (Albert
	C. Sundberg. r.. in The O l d Testanlent d t h c E a r l y Church, Harvard Theological Studies, 1%4). However. a commis-
	sion appointe by a succeeding pope, Innocent VIII, condemned at least ten of Pico's theses as "rash, false, and hereti-
	cal." This seems to have been the attitude of the Church thereafter, the cabala being proscribed as a Jewish system of
	black magic, the "laboratory of Satan."
	40. Tractate Beshallah, Mckilta de Rabbi Ishmael, vol. 1, p. 245.
	41. Martin Buber, Tales of the Hasidim, Later Masters, chapter on Rabbi Yaakov of Sadagora. While God,
	naturally, rejoiced over the victory ofHis Chosen People, He did not like to see His angels crowing over it. Thus, the
	Talmudists describe God as silencing an angelic chorus that chanted hallelujahs when the Egyptian hosts met with
	disaster, by crying out: "How dare you sing in rejoicing when my handiwork [i.e., the Egyptians] is perishing in the
	sea !" [Rf: Ben Zion Bokser, The Wisdom ofthe Talmud, p. 117.1
	}
\end{quote}
	
	
	
	
	
	
	INTRODUCTION
	[xxv]
	Jacob's antagonist at Peniel was God, as Jacob himself finally figured out whcn day broke
	(Genesis 32: 30). But our learned rabbis, after pondering the text, havcconcluded that the antago-
	nist was not God but an angel of God, and that he was either Uriel, Gabriel, Michael, Meta-
	tron, or even Sanlmael, prince of death.42
	When Enoch was translated to Heaven, his angelic guide, according to Enoch's own testi-
	mony, was Uriel. But later on in the same book (Enoch I) Uriel turns out to be Raphael, then Raguel,
	then Michael, then Uriel all over again. Apparently they were the same angel, for Enoch
	throughout speaks of "the angel that was with me." But perhaps it is too much to expect Elloch
	to be consistent. He is, as we have seen, notoriously unreliable. True, we do not have his original
	scripts, or even early copies; the writings accredited to him have come down to us in a hopelessly
	corrupt form, much of it clearly "doctored" to conform to the views of interested parties. Still
	it is hard to believe he was a clear thinker or accurate reporter, although he purports to have
	been an eyewitness in many of the incidents he describes.
	The habitat of angels proved equally perplexing. In the opinion of Aquinas, angels cannot
	occupy two places at the same time (theoretically it would not be impossible for them, being
	pure spirits, to do so). On the other hand, they call journey from one place to another, however
	far removed, in the twinkling of an eye. In angelology, one comes upon instance after instance
	where an angel is a resident of, or presider over, two or three Heavens simultaneously. Thus, in
	Hagiga 12b, Michael is the archistratege of the 4th Heaven. Here he "offers up daily sacrifice."
	But Michael is also governor of the 7th and 10th Heavens. As for Metatron,.he is reputed to
	occupy "the throne next to the throne of Glory," which would fix his seat in the 7th Heaven,
	the abode of God. But we find Metatron, like Michael, a tenant of the 10th Heaven, the primum
	mobile, which is likewise the abode of God-when, that is, God is not in residence in the
	7th.
	Gabriel, lord of the 1st Heaven, has been glimpsed sitting enthroned "on the left-hand side
	of God (Metatron's throne, then, must be on God's right).43 This would indicate that Gabriel's
	proper province is not the 1st but the 7th or 10th Heaven (it was in the 10th Heaven that Enoch
	beheld "the vision of the face of the Lord"). However, according to Milton in Paradise Lost IV,
	549, Gabriel is chief of the angelic guard placed over Paradise, and Paradise being in the 3rd
	Heaven, we should, accordingly, fuld the enthroned Ailnunciator camping out there.
	Logically, one should look for Shamshiel, prince of
	in zebu1 or sagun (the 3rd
	Heaven) where Azrael, suffragan angel of death, lodges, next to the Tree of Life. But some
	
\begin{quote}
		{\small 42. There are any number of princes or angels of death. Prominent among them, besides Sarnrnael, are Kafziel,
	Kezef, Satan, Suriel, Yehudiam, Michael, Gabriel, Metatron, Azrael. AbaddonlApollyon. They are all under orders
	from God. When they fail to accomplish their mission, as in the case of Moses who refused to ive up the ghost, then
	God Himself a m as His own angel of death. According to legend (Ginzberg, The Legends heJews III, 473). after
	God used His best arguments to persuade the aged Lawgiver that he would be better off dead than alive, and the
	Lawgiver still proving stubborn, God descended from Heaven (in the company of Michael, Gabriel, and Zagzagel)
	and "took Moses' soul with a kiss" (Jude 9). The legend goes on to say that God then buried Moses, but "in a spot
	that remained unknown, even to Moses himself."
	43. It is here also lion the right hand of God the Father Almighty" that Jesus sits, according to the Apostles'
	Creed.
	44. Other princes of Paradise include Johiel, ikphon, Zotiel, Michael, Gabriel.
	}
\end{quote}
	
	
	
	
	[xxvi]
	INTRODUCTION
	sources place Shamshiel in charge of the 4th Heaven (also called ~ e b u l ) . ~ On
	' the other hand, if
	we go by The Book oflubilees, Shamshiel is chief of the Watchers, and so properly he would be
	overseeing the 2nd or 5th Heaven, where the Watchers dwell, "crouched in everlasting despair."
	Furthermore, in the guise of Shemuiel (the archonic warden who stands at the windows of
	Heaven "listening for the songs of praise ascending from synagogues and houses ofstudy below"),
	Shamshiel would be posted at the portal of the 1st Heaven. Which leaves Shamshiel where?
	Obviously, in an emergency, it would be difficult to locate him.
	A final instance : Zagzagel or Zagzagael, prince of the Torah, "angel with the horns of glory,''
	is the celestial guard of the 4th Heaven-let us bear in mind that Shamshiel is already in charge
	at this level-and Zagzagel, being at the same time seneschal of the 7th Heaven, his stewardship
	of the 4th Heaven poses a knotty problem. Confusion without end ! One is constrained to cry
	out, with the dying Goethe: "More light !"
	A contemporary of the great Hillel, Ben Hai Hai (identified with another noted rabbi of
	his day, Ben Bag Bag) used to say: "According to the labor is the reward."46 Goethe in Faust,
	part 6, comforts his readers with a similar maxim: "Kiihi das Miihen, herrlich der Lohn"-
	"Daring the labor, lordly the reward."
	If there is any reward for the labor of compiling this Dictionary, it is in the knowledge that
	every effort has been made to keep the sins of con~mission and omission down to a minimum
	(and no one knows better than the author how many sins may be committed in the course of
	such a work). There are still many problems left unresolved here. This is due either to the
	inaccessibility of much of the extant material in the field or to its indecipherability, or because
	the wit and wisdom to provide the solutions were wanting. Future investigators, better equipped,
	for whom some of the underbrush has been cleared away, may be able to provide the solutions,
	along with the names of additional angels that no doubt will turn up in new finds. I might
	interpose here (to paraphrase Rabbi Nathan's famous dictum, "He who preserves a life preserves
	a world") that the preserving of a single angel-not one of the "suffix" creatures-is like pre-
	serving a whole hierarchy. The task certainly is not an easy one, but it may prove easier than
	the one confronting this voyager when he started out on his quest, primed with only the
	scantiest notion of the labor that lay ahead.
	A good way to conclude this Apologia pro libro suo is to quote from a recently published
	paper on the guise of angels. It was there intimated that "in view of the continuing hold of the
	supernatural over the minds of men, and the fact that a belief in the existence of angels (and
	demons) is an article of faith with two of our major world religions, and part of the tradition
	of at least four of them (Persian, Jewish, Christian, Muslim), it is highly probable that we shall
	have the winghd creatures with us for a long, long time to come." True, we may not always
	know whether we are in the presence of "a spirit of health or goblin damned," whether we are
	being fanned by "airs from Heaven or blasts from Hell," but it is best to be on guard. Even
	Satan, as Paul cautioned us, can show himself transformed into an angel of light.
	
\begin{quote}
		{\small 45. In Peter de Abano, Heptumeron, zebu1 is also a designation for the 6th Heaven.
	46. Pirke Aboth, chapter 5, mishna 26.}
\end{quote}
	
	
	
	
	ACKNOWLEDGMENTS
	In the course of compiling this Dictionary, I availed myself of the counsel, knowledge, and help of a host
	of friends. Some read early versions of the text; others were generous with the loan of books; still others
	brought to my attention sources of information I might otherwise not have known. To all such, my
	gatitude and thanks. Appreciating the fact that a list of persons to whom one is indebted is hardly ever
	complete, I ask indulgence of those whose names are here omitted, not through any conscious act of
	mine, but by virtue of a faulty memory-a malady from which, I gather, many human beings suffer.
	From almost the beginning, two scholars encouraged and sustained me; also, on occasion, rescued
	me from exegetical pitfalls: Dr. Harry M. Orlinsky, professor of Bible at Hebrew Union College-
	Jewish Institute of Religion, New York, and Dr. Abraham Berger, chief of the Jewish Division, New
	York Public Library. In acknowledging my indebtedness to these distinguished colleagues and friends,
	I absolve them at the same time of responsibility for any errors, oversights, theological sins, indefensible
	assumptions or conclusions of which I may be guilty, and which are apt to occur in a work of this kind
	and extent, despite every effort at rooting them out. The responsibility is solely mine. I cheerfully
	shoulder it. And I leave it to Hamlet's "angels and ministers of grace" to defend me.
	In the Oriental Division of the New York Public Library where I was (and still am) a frequent
	visitor, I benefited greatly from the friendly interest and wide-ranging knowledge of Francis Paar and
	Zia U. Missaghi (Ray Lord). They were unsparing of their time and help. In the Rare Book Room and
	in the quarters of the Berg Collection at the same institution I found the directors and the staff members
	equally knowledgeable, obliging, and helpful.
	Gershom Scholem of the Hebrew University, Jerusalem, in response to my inquiries as to the identity
	of the right and left emanations of God (the sefiras), generously provided me with their names, along
	with the sources where I might come upon them (the 16th-century texts ofJacob and Isaac ha-Cohen of
	Soria). I am extremely grateful to Dr. Scholem. I am grateful to Dr. Solomon Zeitlin of Dropsie College,
	Philadelphia, for trying to "authenticate" the seven archangels "that stand and enter before the glory of
	the Lord" (The Book of Tobit). I am indebted to Prof. Theodor H. Gaster of Columbia University for
	his interesting observations on the angel Suriel; and to Prof. Bruce M. Metzger of Princeton for making
	clear his views on Jeremiel and Uriel as being the same angel under different names. I am equally under
	obligation to Dr. Meir Havazelet of Yeshiva University, New York, who culled angels for me from the
	minor midrashim and who did not hesitate to ring me up in the middle of the night to spell out the
	names of winged creatures he had suddenly come across (in hechaloth or Merkabah lore) and which, he
	feared, I might have overlooked.
	
	
	
	
	
	
	[xxviii]
	ACKNO WLEDCMENTS
	I would be remiss if I did not speak here of the help accorded me by the late H. D. (Hilda Doo-
	little), noted American poet long resident abroad. She was an avid reader in esoterica; also a devout
	believer in angels, whom she invoked by name and apostrophized in song. From Zurich, where she made
	her home for many years until her recent death, she sent me rare books in practical cabala "for our
	mutual benefit." Our friendship, though brief and late in coming, I count among my most cherished
	memories.
	perhaps this would be a good place to make general acknowledgment to editors, authors, pub-
	lishers, heads of libraries and museums, custodians or owners of works of art, for permitting the use of
	illustrations over which they hold the right of reproduction. Specific acknowledgment is made through-
	out the Dictionary where such illustrations appear. And, for their friendly cooperation, help, patience,
	and indulgence, I am happy to record my gratitude to the editorial and production staffs of The Free
	Press and The Macmillan Company.
	This would be a good place also to speak of the unwavering interest, devotion, and faith in my
	work on the part of my wife Mollie, who proved to be at all times my severest critic (hence, my
	best friend). T o her I owe and acknowledge a debt of gratitude which I know I shall never be able
	fully to discharge.
	And now, without interruption, a roster of those many others who, over the years, in greater or
	lesser degree, and perhaps without themselves being aware of having done so, enlivened and enhanced
	my labors, if only through a chance remark, an apt quotation, the verification of a date or the title of a
	book. Here then they are, from A to Y:
	
	{\small John Williams Andrews, Professor Charles Ango6 Oscar Berger, Rabbi Ben Zion Bokser,
	Josephine Adams Bostwick, Edmund R. Brown, Eric Burger, Vera and Eduardo Cacciatore, Herbert
	Cahoon, Leo Cherne, Thomas Caldecot Chubb, Frank E. Cornparato, Miriam Allen De Ford, Eugene
	Delafield, Arto DeMirjian, Jr., Dr. Alfred Dorn, Alexis Droutzko~, Dan D d m , Richard Ellis, Prof.
	Morton S. Enslin, John Farrar, Emanuel Geltman, Dr. Jivko Ghelev, Louis Ginsberg, Dean ~ o y d
	Haberly, the late Prof. Moses Hadas, Geoffrey ~ a n d l e ~ - T a ~ l o
	Hector
	r,
	Hawton, Prof. Abraham
	Joshua Heschel, Richard Hildebrand, Calvin Hoffman, Arthur A. Houghton, Jr., James Houston, W .
	Carter Hunter, Sulamith Ish-Kishor, Jeremiah Kaplan, Abraham Eli Kessler, John Van E. K o h , Surya
	Kumari, Myra Reddin Lalor, Isobel Lee, Dr. Elias Lieberman, Dr. Gerhard R. Lomer, Eugenia S. Marks.
	Prof. Alfeo Marzi, Samuel Matza, Edward G. McLeroy, Gerard Previn Meyer, arth ha Mood, Prof.
	Harry Morris, Kay Nevin, Rabbi Louis I. Newman, Louise Townsend ~ i c h o l l Hugh
	,
	Robert Orr, Jane
	s,
	Blaffer Owen, Mrs. Lori P. Podesta, Jane Purfield, Prof. Joseph Reider, Mrs. R. S. ~ e ~ n o l d Sr.,
	Rossell Hope Robbins, Leighton Rollins, Liboria Romano, Sylvia Sax, ~ o w a r d Sergeant, ~ o b e r t
	Sargent Shriver, Jr., Isaac Bashevis Singer, Chard Powers Smith, the late prof. Homer W. smith,
	Sidney Solomon, Prof. Walter Starkie, Rabbi Joshua Trachtenberg, prof. ~ o s e Tusiani,
	~h
	valev ebb,
	Charles A. Wagner, Vivienne Thaul Wechter, Prof. Robert H. West, oh all ~ h e e l o c k ~stelle
	,
	Whelan, Basil Wilby, Claire Williams, Prof. Harry A. Wolfson, Dr. Amado M. Yuzon.
	
	
	}
	
	
	
	
	
	
	
	
	
	
	
	
	
	
	
	
	
	
	
	
	
	
	
	
\end{document}